The computational science and engineering (CSE) communities have a mixed record
of using software engineering and adopting good software
practices. Many codes adopt software practices when the size
and composition of the code makes it impossible to make progress
without them. In rarer instances code projects start with an awareness
of the importance of software process and build it in from the
beginning. As more codes have crossed the threshold of being
manageable without software engineering they have increasingly been  
adopting software processes derived from outside the scientific
domain. The driving force behind adoption is usually the realization
that without using software engineering practices, the development,
verification and maintenance of code becomes
intractable. State-of-the-art software for software engineering
practices in CSE codes lags behind that in the software
industry. There are many reasons for it, but the primary one is that
not all software engineering practices can be adopted by the CSE code
developers. 

many software best practices are not well-suited for CSE codes without
modification and/or customization, in particular to multiphysics
multicomponents codes that run on the largest HPC platforms. Sometimes
the inherent physics of scientific applications require different
software methodologies, at others a premium is placed on
performance rather than code architecture.  Still others are more
sociological and due to the type of institutions where such codes are
developed. The challenges for scientific applications range
from their architecture to the process for their maintenance and
growth. The standard practices adopted by the CSE codes include
repositories for code version control, modular code design, licensing
process, regular testing, documentation, release and distribution
policies and contribution policies. Less frequently used practices
include code-review and code-depracation. Agile developmenet methods
tend to have more limited usefulness in the CSE lifecycle, only a
select subset of those are used. The degree of adoption and
sophistication in using these  practices varies among teams. Many of
the reasons behind less penetration are discussed in section
\ref{sec:institutional-challenges}. Almost all the software
engineering practices mentioned above have to be modified and
customized for CSE software. The next two sections outline the
challenges that are either unique to, or are more dominant in this
domain than elsewhere. 


% For example,
% sometimes modularity and encapsulation principles are challenged by
% the need to tightly couple physics solvers to data structures.
% Scientific codes are designed to explore phenomena that are not very
% well understood, so their verification strategies have  to
