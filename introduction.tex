The computational science and engineering (CSE) communities have a mixed record
of using software engineering and adopting good software
practices. Majority of codes adopt software practices when the size
and composition of the code makes it impossible to make progress
without them. In rarer instances code projects start with an awareness
of the importance of software process and build it into the DNA of the
code. As more codes have crossed the threshold of being manageable
without software engineering they have increasingly been 
adopting software processes derived from outside the scientific
domain. The driving force behind adoption is usually the realization
that without using software engineering practices, the development,
verification and maintenance of code becomes intractable. However,
many software best practices are not well-suited for CSE codes without
modification and/or customization. Sometimes the inherent physics of
scientific applications require different software methodologies,
while other times a premium is placed on performance rather than
code architecture.  Still others are more sociological and due to the
type of institutions that are the typical homes for such codes. The
challenges for scientific applications range from their architecture
to the process for their maintenance and growth. 
% For example,
% sometimes modularity and encapsulation principles are challenged by
% the need to tightly couple physics solvers to data structures.
% Scientific codes are designed to explore phenomena that are not very
% well understood, so their verification strategies have  to
