\label{sec:introduction}
Computational science and engineering (CSE) communities develop
complex applications to solve scientific and engineering challenges,
but these communities have a mixed record of using software
engineering best practices. Many codes developed by CSE communities
adopt standard software practices when the size and complexity of an
application becomes too unwieldy to continue without them
\cite{cc2012}. The driving force behind adoption is usually the
realization that without using software engineering practices, the
development, the verification and the maintenance of applications can become
intractable. As more codes cross the threshold into increasing
complexity, software engineering processes are being adopted from
practices derived outside of the scientific and engineering domain.
State-of-the-art for software engineering practices in CSE codes often lags
behind that in the commercial software space
\cite{segal2008developing,basili2008understanding, hochstein2008asc}. 
% \comment{KA NOTE: Do we have any
% reference for this claim? AD: I don't know of one, though this seems
% to be the consensus.  Let's just soften a little then.  I added "often"}
There are many reasons for it, first, is
that not all software engineering practices can actually be adopted by
the CSE code developers.  Secondly, there is often little funding for
code development and maintenance. And finally, there is very little
research in the software engineering community targeted to specific
needs of CSE codes. This chapter elaborates on the above challenges and
how they were addressed in FLASH and Amanzi, two codes with very
different develepment timeframe and therefore very different
development paths. 

FLASH was originally designed for computational
astrophysics in the late 1990's. It has been almost continuously been
under production and development since 2000 with three major
revisions. It has exploited an extensible framework to expand its
reach a community code for over half a dozen scientific
communities. The software engineering practices for the code have
evolved with time, and their adoption has grown with the complexity of
the code. Amanzi, on the other hand, started in 2012 and has developed
from the ground up in C++ using relatively modern software engineering
practices. It still has one major target community, but is also
designed with extensibility as an objective. There are many other
similarities and some difference described later in the chapter.
In particular, we address the issues related to software
architecture and modularization, design of a testing regime,
unique documentation needs and challenges, use of versioning system 
for managing projects, and the tension between intellectual property
management and open science.

Many software best practices are not well-suited for CSE codes without
modification and/or customization.  In particular, multiphysics and
multicomponents codes that run on the large High Performance Computing
(HPC) platforms have specific requirements.  In some cases, the
inherent physics of scientific applications require different software
methodologies.  In other cases, a large premium is placed on
performance, rather than code architecture, making it necessary to
sacrifice some known software engineering best practices.  Still
others are more sociological because codes may be developed by domain
scientists and their graduate students who have different priorities
and scientific goals.  
%and due to the type of institutions where
%such codes are developed.  
The challenges in developing scientific and engineering applications range
from data dependencies, to architectural trade-offs, to the process for their maintenance and
growth. 
% \comment{KA NOTE: see if you like what I re-wrote in last sentence.}
The standard practices adopted by the CSE codes include
repositories for code version control, modular code design, licensing
process, regular testing, documentation, release and distribution
policies, and contribution policies \cite{cc2012, carver2012software,
Dubey2014}. Less frequently used practices include code-review,
code-deprecation, and adoption of development methodologies such as
Agile \cite{agile}.  
%Agile development methods \cite{}
%tend to have limited usefulness in the CSE lifecycle.
% only a 
%select subset of those have met with success.  (KA NOTE: NEED REF) 
The degree of adoption and sophistication in using software
engineering practices varies among teams. Many of the reasons for
lower penetration of more formal software engineering practices are
discussed in section \ref{sec:instChallenges}. Even among the widely
adopted practices, most are modified and customized by the developers
for their own needs. The next few sections outline the challenges that  
are either unique to, or are more dominant in this domain than
elsewhere. This is followed by a section describing experiences of
FLASH and amanzi development teams. FLASH belongs in the first generation of
codes that adopted a software process. This was in the era when the
advantages of software engineering were almost unknown in the
scientific world. Amanzi is from the ``enlightened'' era (by
scientific codes standards) where a minimal set of software practices are
adopted by most code projects intending long term use. A study of
software engineering of two codes from differen different
eras of scientific software development highlight how these practices
and the communities have evolved.   


% For example,
% sometimes modularity and encapsulation principles are challenged by
% the need to tightly couple physics solvers to data structures.
% Scientific codes are designed to explore phenomena that are not very
% well understood, so their verification strategies have  to
