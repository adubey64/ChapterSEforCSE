\documentclass[11pt]{article}

\usepackage{natbib} 
\bibpunct{[}{]}{;}{n}{,}{,} % square brackets
% "Greater/Less than or of order" symbols
\def \la {\mathrel{\vcenter
     {\hbox{$<$}\nointerlineskip\hbox{$\sim$}}}}
\def \ga {\mathrel{\vcenter
     {\hbox{$>$}\nointerlineskip\hbox{$\sim$}}}}

\textwidth 6.1in
\textheight 8.25in
\oddsidemargin 0.24in
%\evensidemargin -0.5in
\topmargin -0.25in


\begin{document}
%\maketitle
\noindent{\center \large Proposal for a Chapter in Section 1}\\ 

\noindent{\large{\bf Authors:}} Anshu Dubey (CRD, LBNL), Katherine
Riley(ALCF, ANL) and Katie Antypas(NERSC, LBNL) \\
The authors of the proposed chapter were instrumental in developing
the software design, process and policies for FLASH. The FLASH code
has been a successful community code for Astrophysics for nearly 15
years and  several other communities for the past several years. 
Anshu Dubey lead the team developing FLASH for over a decade, and
oversaw 2 major version releases. She has several successful
workshops on community codes, and software design for future.
Katie Antypas heads the Computing and Data Services Department at
NERSC, which includes the Advanced Technologies, Data and Analytics
Services, Storage Systems, and User Services Groups. The Department
works to increase the impact of science research conducted by NERSC
users by providing exceptional services and user support, preparing
users for future architectures, and providing comprehensive services to
store, analyze, manage and share data. Katherine Riley is the Deputy Science
Director for ALCF where her responsibilities include leading the team
of computational scientists (Catalysts). The team collaborates with
scientists using the largest supercomputers for their research by
offering expertise in scalable algorithms, performance, and workflows.
Katie and Katherine have dealt  with a wide variety of application
codes, and have had close interactions with several groups about their
software engineering needs and challenges. Katherine and Anshu
also lead and organize the ``Software Engineering and Community
Codes'' track in the Argonne Training Program for Extreme Scale
Computing (ATPESC). \\ 


\noindent{\large \bf Title : Software Process for Multicomponent Multiphysics
  Codes}\\ \\
\noindent{\large \bf Abstract: }
Multiphysics multicomponent scientific codes have increasingly been
adopting software processes derived from outside the scientific
domain. The driving force behind adoption is usually the realization
that without using software engineering practices, the development,
verification and maintenance of code becomes intractable. However,
many software best practices need modification and/or customization.  Sometimes the inherent physics of scientific applications require different software methodologies, while other times a premium is placed on performance rather than architecture.  Still others are more sociological and due to the
type of institutions that are the typical homes for such codes. The
challenges for scientific applications range from their architecture to the process for their
maintenance and growth. For example, sometimes modularity and
encapsulation principles are challenged by the need to tightly couple physics solvers to data structures.  Scientific codes are
designed to explore phenomena that are not very well understood, so
their verification strategies have to be customized. The
components of scientific codes must also consider accuracy and stability, and
interoperability between components must not
violate physical constraints. These constraints are particularly challenging for
numerical software because a wrong answer may not be detected to be wrong
because of the exploratory nature of the problems being addressed.  Finally, as
the research in the field evolves, codes may have to undergo radical
changes in base algorithms to include the newly acquired knowledge. And
the metrics of success are harder to define. 

The institutional challenges in developing and maintaining scientific codes arise because the 
research institutions in which they are developed often have an unreliable
funding model and a transient developer population of graduate students. There are
rarely enough resources  to have any redundancy in expertise which makes
procedures like code review difficult. Intellectual ownership, code
distribution, attribution of credit and contribution policies can all
become thorny issues because there is often a lack of communication
and sometimes even distrust among the stakeholders. Documentation
takes on a different kind of urgency because of transient expertise,
and yet it is notoriously difficult to persuade the 
developers in the field to spend their limited time providing
exhaustive documentation. 

We propose a chapter which will elaborate on the above challenges and
how they were addressed in FLASH, a code that has been extremely
successful as a community code for several research communities. We
will outline specific solutions used by FLASH, and discuss their
possible generalizations that are usable by other similar software
efforts. In particular, we will address the issues related to software
architecture and modularization, design of a testing regime,
unique documentation needs and challenges, use of versioning system 
for managing projects, and the tension between intellectual property
management and open science. \\


\noindent{\large \bf Chapter Outline}
\begin{itemize}
\item {\bf Introduction} : this section will give an overview of what
  this chapter is about
\vspace {-0.1in}
\item{\bf Domain Challenges} : this section will outline challenges
  that are unique to scientific codes because of the nature of
  problems they try to solve
\vspace {-0.1in}
\item {\bf Institutional Challenges}: this section will talk about
  challenges that come about because of the kind of institutions in
  which these codes are developed
\vspace {-0.1in}
\item {\bf Case Study} : How they were addressed by the FLASH team
\begin{itemize}
\vspace {-0.1in}
\item In the code design
\vspace {-0.1in}
\item In the software process
\vspace {-0.1in}
\item in the policies
\end{itemize}
\vspace {-0.1in}
\item{\bf Generalization} : this section will discuss those aspects of
  FLASH solutions that are generalizable
\vspace {-0.1in}
\item{\bf Additional Future Considerations}: How the software,
  design and policies might need to change in Future.
\end{itemize}

\noindent{\large \bf Justification For the Chapter:}
The software engineering practices outlined in the chapter have been
in use in almost all the successful community codes in various
scientific domains. All communities arrived at what looks like
remarkably similar solutions as we found in an NSF funded workshop on
community codes held at the University of Chicago in 2012. Even the
modifications to the software engineering practices were similar. This
content has also been included in the ``Software Engineering and
Community Codes'' track in ATPESC, the two week training program that Argonne National
Laboratory has been conducting for the past two years. The students
have been extremely appreciative of the track and have found it to be
useful in their work. Therefore we believe that this chapter will be
of real practical use to the domain scientists and developers of
scientific software.
\end{document}




