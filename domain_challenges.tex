\label{sec:domain-challenges}
Multi-physics codes, by their definition, solve more than one
mathematical model. A typical code combines 3-4
diverse models, in extreme cases codes may employ many more.
Rarely, all models work with the same
discretization using similar algorithmic approach (for instance
stencil computations in explicit PDE solvers).
Models with diverse discretizations and algorithms are more common. In
such cases, each operator has
its own preferred data layout and movement that may differ
from those needed by other operators. Different models may also need
different developer expertise. Therefore, when designing scientific software,
it is especially important to create  modular components wherever
possible.  This is because different expertise may be required to
understand each component, and thus a modular design allows
application developers to focus on the areas they know best. For
example, the numerical algorithms associated with physics operators
require mathematical expertise, which is different from the code
architecture which requires software engineering expertise. In
addition, modular code allows various components to interface with
each other in a clearer way.  

Though desirable, modularization can also be difficult to achieve in
multi-model codes. Normally challenges related to different data
layout and memory and communication access patterns 
can be mitigated through encapsulation and well defined APIs. The outer
wrapper layers of the operators can carry out data transformations as
needed. There are two factors that work against this approach in scientific
codes. One is that scientific simulation codes model the physical world which does not
have neat modularization. Various phenomena have tightly coupled
dependencies that are hard to break. These dependencies and tight
couplings are transferred into the mathematical models and hinder the
elimination of lateral interactions among code modules. An attempt to
force encapsulations by hoisting up the lateral interactions to the
API level can explode the size of the API. And if not done carefully
this can also lead to extra data movement. The second is that 
the codes are performance sensitive and wholesale data movement
significantly degrades performance. 
The module designs, therefore, have to be cognizant of potential
lateral interactions and make allowances for them. Similarly, the data
structures have to take into account the diverse demands placed on
them by different operators and carefully consider the trade-offs
during software design. Considerations such as these are not common in
software outside of scientific domains.    
Section \ref{sec:case-studies} describes how these challenges have
been met by FLASH and Amanzi.

Multi-physics multi-scale codes often require tight integration with
third party software, which comes in the form of numerical
libraries. Because multi-physics codes combine expertise from many
domains, the numerical solvers they use also require diverse applied
mathematics expertise. It can be challenging  for any one team to
assemble all the necessary expertise to develop their own software and
so many turn to third party math libraries for highly optimized
routines.  However, as mentioned in section \ref{sec:using}, the use
of third party software does not absolve them from understanding its
appropriate use.  Additionally, information about appropriate use of
third party software within the context of a larger code must also be
communicated to the users of the code. 

\subsection*{Key Insights}
\label{domain-insights}
\begin{itemize}
\item Multi-physics codes need modularization for separation of
concerns, but modularization can be hard to achieve because of lateral
interactions inherent in the application.
\item Codes can use third party libraries to fill their own expertise
gap, but they must understand the characteristics and limitations of
the third party software 
\end {itemize}