%\begin{itemize}
%item diverse algorithms - different data layout needs
\label{sec:domain-challenges}
Multiphysics codes, by their definition, solve more than one
mathematical model. A typical code combines 3-4
diverse models, in extreme cases codes may employ many more.
%\comment[(KA NOTE: NEED REF)]  
Rarely, all models work with the same
discretization using similar algorithmic approach (for instance
stencil computations in explicit PDE solvers).
Models with diverse discretizations and algorithms are more common. In
such cases, each operator has
its own preferred data layout and movement that may differ
from those needed by other operators. Different models may also need
different developer expertise. Therefore, when designing CSE software,
it is especially important to create  modular components wherever
possible.  This is because different expertise may be required to
understand each component, and thus a modular design allows
application developers to focus on the areas they know best. For
example, the numerical algorithms associated with physics operators
require mathematical expertise, which is different from the code
architecture which requires software engineering expertise. In
addition, modular code allows various components to interface with
each other in a clearer way.  % There is another concern for CSE codes 
% that use large HPC systems and require code parallelization.  Parallel
% codes typically must implement a parallel domain decomposition and
% manage synchronization and load balancing between tasks or threads.

Though desirable, modularization can also be difficult to achieve in
multi-model codes. Normally challenges related to different data
layout and memory and communication access patterns 
can be mitigated through encapsulation and well defined APIs. The outer
wrapper layers of the operators can carry out data transformations as
needed. There are two factors that work against this approach in CSE
codes. One is that CSE simulation codes model the physical world which does not
have neat modularization. Various phenomena have tightly coupled
dependencies that are hard to break. These dependencies and tight
couplings are transferred into the mathematical models and hinder the
elimination of lateral interactions among code modules. An attempt to
force encapsulations by hoisting up the lateral interactions to the
API level can explode the size of the API. And if not done carefully
this can also lead to extra data movement. The second is that 
the codes are performance sensitive and wholesale data movement
significantly degrades performance. 
% Parallelization models and management of
% load distribution are additional class of concerns that the codes
% using HPC platforms face. % and  With appropriate code
% modularization that enables separation of concerns these different
% aspects of the code do not need to interfere with one another and can
% help make application development more tractable. 
The module designs,
therefore, have to be cognizant of potential lateral interactions and
make allowances for them. Similarly, the data structures have to take
into account the diverse demands placed on them by different operators
and carefully consider the trade-offs during software
design. Considerations such as these are not common in software
outside of CSE.   
Section \ref{sec:case-studies} describes how these challenges have
been met by FLASH and Amanzi.


%\item physics is messy - encapsulation can be challenging
%\comment[(KA NOTE: NEED REF - can we back up this claim?)}

%\item less logical complexity more numerical complexity - hard to achieve data locality 

% In a CSE software design, separation of concerns, where different
% expertise addresses different aspects of code behavior, is of utmost
% importance.
% have little or no correlation with the numerical algorithm
% implementation. 
%\comment[(KA NOTE:  I don't know what this last sentence means.  Do you mean modular design is important?)}
% This orthogonalization of expertise requirements into different code
% components 
% that allows developers to focus on what they know best.  
% Another natural fallout of this approach is that different dimensions of
% complexities in the algorithm space are handled separately. 
% They are not logically as complex. Whereas machinery for
% managing the discretizations and interoperability among code
% components is likely to be less complex numerically, but could be very
% complex logically. 
% \comment[(KA NOTE:  I don't think logically complex and accuracy complex are well known terms.  I think the point we're trying to make isn't coming through here.  Is it important?)}
% A third axis of concern is parallelization, which
% brings in some features that are unique to CSE codes, such as domain
% decomposition, aspects of synchronization and dependencies, and
% performance impact of the design choices.
% With appropriate separation of concerns not only do these aspects of software development not 
% interfere with one another, they help make the development tractable. 

% \comment{(KA NOTE: I re-wrote the paragraph above to the lines below.  Let me know what you think.  My thought is to remove references to 'separation of concerns' which I think is vague and not well known in the community.)}


%\item need for third party software 
Multiphysics multiscale codes often require tight integration with third party software, which comes
in the form of numerical libraries. Because multiphysics codes combine
expertise from many domains, the numerical solvers they use also
require diverse applied mathematics expertise. It can be challenging 
for any one team to assemble all the necessary expertise to develop their own software and so many turn to third party math libraries for highly optimized routines.  However, as mentioned in section \ref{sec:using}, the use of
third party software does not absolve them from understanding its appropriate
use.  Additionally, information about appropriate use of third party
software within the context of a larger code must also be communicated
to the users of the code. 
% And finally, with the addition of third
% party software into an application, the team is now dependent on an
% outside party for optimizations and porting to new platforms. 

%\item unit testing insufficient, not even always possible
% Some of these issues are addressed in the chapter on testing,
% we can perhaps cross reference.
%\item integrated and system level testing very critical
%\item robustness and stability as important as accuracy


%\end{itemize}

