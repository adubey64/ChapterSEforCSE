%\begin{itemize}
%item diverse algorithms - different data layout needs
\label{sec:domainChallenges}
Multiphysics codes, by their definition, have more than one
mathematical model that they are solving. A typical code combines 3-4
diverse models, the more extreme ones may employ as many as a
dozen. (KA NOTE: NEED REF)  In a rare calculation all models work with the same
discretization using similar algorithmic approach (for instance
stencil computations in explicit PDE solvers). More common is to have
models with diverse discretizations and algorithms. In this case, each operator has
its own preferred data layout and movement, and it usually differs
from those needed by the other operators.  Normally these challenges
can be mitigated through encapsulation and well defined API's. The outer
wrapper layers of the operators can carry out data transformations as
needed. There are two factors against taking this approach in CSE
codes: (1) physics is not always friendly to encapsulation, and (2)
the codes are performance sensitive and wholesale data movement
significantly degrades performance. 

%\item physics is messy - encapsulation can be challenging
The CSE simulation codes model the physical world which does not
have neat modularization. Various phenomena have tightly coupled
dependencies that are hard to break. These dependencies and tight
couplings also translate into their mathematical models and it becomes
hard to eliminate lateral interactions among code modules implementing
the models. An attempt to force encapsulations by hoisting up the
lateral interactions to the API level can explode the size of the
API. And if not done carefully this can also lead to extra data
movement. The module designs, therefore, have to be cognizant of
potential lateral interactions and make allowances for them.
Similarly, the data structures have to take into account the diverse
demands placed on them by different operators and carefully consider
the trade-offs during software design. Considerations such as these
are not common in software outside of CSE.  (KA NOTE: NEED REF - can we back up this claim?)

%\item less logical complexity more numerical complexity - hard to achieve data locality 

In a CSE software design, separation of concerns is of utmost
importance. (KA NOTE:  I don't know what this last sentence means.  Do you mean modular design is important?)
Orthogonalization of expertise requirements into different code
components allows developers to focus on what they know best.  
Another natural fallout of this approach is that different dimensions of
complexities in the algorithm space are handled separately. The
numerical algorithms associated with physics operators are complex
because of accuracy and stability concerns, and require mathematical
expertise. They are not logically as complex. Whereas machinery for
managing the discretizations and interoperability among code
components is likely to be less complex numerically, but could be very
complex logically. (KA NOTE:  I don't think logically complex and accuracy complex are well known terms.  I think the point we're trying to make isn't coming through here.  Is it important?)  A third axis of concern is parallelization, which
brings in some features that are unique to CSE codes, such as domain
decomposition, aspects of synchronization and dependencies, and
performance impact of the design choices. With appropriate separation
of concerns not only do these aspects of software development not 
interfere with one another, they help make the development tractable. 

(KA NOTE: I re-wrote the paragraph above to the lines below.  Let me know what you think.  My thought is to remove references to 'separation of concerns' which I think is vague and not well known in the community.)
When designing CSE software, it is especially important to create modular components wherever possible.  This is because in depth applied math and algorithms expertise may be required to understand each component, and thus a modular design allows application developers to focus on they areas they know best.  In addition, modular code allows various components to interface with each other in a clearer way.  Another challenge for CSE codes is that they run on large HPC systems and require code parallelization.  Parallel codes typically must implement a parallel domain decomposition and manage synchronization and load balancing between tasks or threads.  With appropriate code modularization, these different aspects of the code, do not need to interfere with one another and can help make application development more tractable.


%\item need for third party software 
Multiphysics multiscale codes are often require tight integration with third party software, which comes
in the form of numerical libraries. Because multiphysics codes combine
expertise from many domains, the numerical solvers they use also
require diverse applied mathematics expertise. It can be challenging 
for any one team to assemble all the necessary expertise to develop their own software and so many turn to third party math libraries for highly optimized routines.  However, as mentioned in section \ref{sec:using}, the use of
third party does not absolve them from understanding its appropriate
use.  Additionally, information about appropriate use of third party
software within the context of a larger code must also be communicated
to the users of the code.  And finally, with the addition of third party software into an application, the team is now dependent on an outside party for optimizations and porting to new platforms.

%\item unit testing insufficient, not even always possible
% Some of these issues are addressed in the chapter on testing,
% we can perhaps cross reference.
%\item integrated and system level testing very critical
%\item robustness and stability as important as accuracy
(KA NOTE: consider bringing this testing section, and section below on application portability up to be a section behind 1.2.3 Maintenance and Extensions.  Both testing and application portability seem like they would do better under Lifecycle (and are more general) rather than keeping them under section 1.3 Domain Challenges.)
Testing of CSE software needs to reflect the layered complexity that
the codes themselves have. The first line of attack is to develop the unit tests which isolate testing of one component of the code.
However, as mentioned in chapter \ref{chp: }, in CSE codes, often there are dependencies between different components of the code that can not be isolated, making unit testing more difficult. In these cases, testing should be done with a minimal possible combination of
components.  In effect, these
minimally combined tests play the same role in the testing regime that
unit tests do because they focus on possible defects in a very narrow
section of the code. In addition, multicomponent CSE software should test various 
permutations and combination of components in different ways. Configuring tests in this manner will help verify
that all configurations are within the accuracy and stability
constraints.  

%\item performance portability important
(KA NOTE: consider expanding and making own section.  KA can work on it.  Consider bringing up a level to follow section 1.2.3 Maintenance and Extensions)
Another aspect of multiphysics CSE software is its need for
performance portability. HPC machines are expensive and rare resources and in order to achieve high application performance, codes need to be optimized for the unique HPC architectures.
 However, typical lifecycle of a
multiphysics application spans many generations HPC systems which have a typical lifespan of about 4-5 years.  Depending upon the size of the
code, optimization for a specific target platform can take a
significant fraction of the platform lifecycle, time when the code may not be available for science runs.  Without careful planning and coordination,  a
large fraction of scientists' time could be lost in porting and optimizing
a code for a new system.  Developers have the choice of adding machine specific optimizations or creating more general optimization that will work for a broader class of systems.  CSE codes must consider the trade-offs and advantages of a highly optimized code for a single platform or a design 
HPC CSE codes consider the trade-offs and opt to design their software
using constructs that perform modestly well across a range of
platforms. 


%\end{itemize}

