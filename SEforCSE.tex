\documentclass[11pt]{article}

\usepackage{natbib} 
\bibpunct{[}{]}{;}{n}{,}{,} % square brackets
% "Greater/Less than or of order" symbols
\def \la {\mathrel{\vcenter
     {\hbox{$<$}\nointerlineskip\hbox{$\sim$}}}}
\def \ga {\mathrel{\vcenter
     {\hbox{$>$}\nointerlineskip\hbox{$\sim$}}}}

\textwidth 6.1in
\textheight 8.25in
\oddsidemargin 0.24in
%\evensidemargin -0.5in
\topmargin -0.25in
\title{Software Process for Multicomponent Multiphysics
  Codes}
\author{A. Dubey, Katherine Riley and Katie Antypas}

\begin{document}
\maketitle

\begin{abstract}
Multiphysics multicomponent scientific codes have increasingly been 
adopting software processes derived from outside the scientific
domain. The driving force behind adoption is usually the realization
that without using software engineering practices, the development,
verification and maintenance of code becomes intractable. However,
many software best practices need modification and/or customization.
Sometimes the inherent physics of scientific applications require
different software methodologies, while other times a premium is
placed on performance rather than architecture.  Still others are more
sociological and due to the type of institutions that are the typical
homes for such codes. The challenges for scientific applications range
from their architecture to the process for their maintenance and
growth. For example, sometimes modularity and encapsulation principles
are challenged by the need to tightly couple physics solvers to data
structures.  Scientific codes are designed to explore phenomena that
are not very well understood, so their verification strategies have to
be customized. % The components of scientific codes must also consider
% accuracy and stability, and interoperability between components must not
% violate physical constraints. These constraints are particularly challenging for
% numerical software because a wrong answer may not be detected to be wrong
% because of the exploratory nature of the problems being addressed.  Finally, as
% the research in the field evolves, codes may have to undergo radical
% changes in base algorithms to include the newly acquired knowledge. And
% the metrics of success are harder to define. 
The institutional challenges in developing and maintaining scientific codes arise because the 
research institutions in which they are developed often have an unreliable
funding model and a transient developer population of graduate
students. 
% There are
% rarely enough resources  to have any redundancy in expertise which makes
% procedures like code review difficult. 
Intellectual ownership, code 
distribution, attribution of credit and contribution policies can all
become thorny issues because there is often a lack of communication
and sometimes even distrust among the stakeholders. 
% Documentation
% takes on a different kind of urgency because of transient expertise,
% and yet it is notoriously difficult to persuade the 
% developers in the field to spend their limited time providing
% exhaustive documentation. 
We elaborate on the above challenges and
how they were addressed in FLASH, a code that has been extremely
successful as a community code for several research communities. We
outline specific solutions used by FLASH, and discuss their possible
generalizations that are usable by other similar software 
efforts. In particular, we address the issues related to software
architecture and modularization, design of a testing regime,
unique documentation needs and challenges, use of versioning system 
for managing projects, and the tension between intellectual property
management and open science. 
\end{abstract}


\section {Introduction} 
The computational science and engineering (CSE) communities have a mixed record
of using software engineering and adopting good software
practices. Majority of codes adopt software practices when the size
and composition of the code makes it impossible to make progress
without them. In rarer instances code projects start with an awareness
of the importance of software process and build it into the DNA of the
code. As more codes have crossed the threshold of being manageable
without software engineering they have increasingly been 
adopting software processes derived from outside the scientific
domain. The driving force behind adoption is usually the realization
that without using software engineering practices, the development,
verification and maintenance of code becomes intractable. However,
many software best practices are not well-suited for CSE codes without
modification and/or customization. Sometimes the inherent physics of
scientific applications require different software methodologies,
while other times a premium is placed on performance rather than
code architecture.  Still others are more sociological and due to the
type of institutions that are the typical homes for such codes. The
challenges for scientific applications range from their architecture
to the process for their maintenance and growth. 
% For example,
% sometimes modularity and encapsulation principles are challenged by
% the need to tightly couple physics solvers to data structures.
% Scientific codes are designed to explore phenomena that are not very
% well understood, so their verification strategies have  to

\section{Domain Challenges} : this section will outline challenges
  that are unique to scientific codes because of the nature of
  problems they try to solve
\section{Institutional Challenges}: this section will talk about
  challenges that come about because of the kind of institutions in
  which these codes are developed
\section{Case Study: The FLASH Code}
\section{Generalization} : this section will discuss those aspects of
  FLASH solutions that are generalizable
\section{Additional Future Considerations}: How the software,
  design and policies might need to change in Future.
\bibliographystyle{natbib}
\bibliography{flash.bib}
\end{document}




