\documentclass[sunil1]{sunil} %See documentation for other class options
%\documentclass[sunil1,ChapterTOCs]{sunil} %See documentation for other class options
\usepackage{fixltx2e,fix-cm}
\usepackage{amssymb}
\usepackage{amsmath}
\usepackage{graphicx}
\usepackage{subfigure}
\usepackage{makeidx}
\usepackage{multicol}
\usepackage{paralist}
\usepackage{color}
\usepackage{hyperref}

\newcommand{\comment}[1]{\textcolor{red}{{\bf#1}}}
\newcommand{\response}[1]{\textcolor{blue}{{\bf#1}}}

\frenchspacing
\tolerance=5000

\makeindex

%\bibpunct{[}{]}{;}{n}{,}{,} % square brackets
% "Greater/Less than or of order" symbols
\def \la {\mathrel{\vcenter
     {\hbox{$<$}\nointerlineskip\hbox{$\sim$}}}}
\def \ga {\mathrel{\vcenter
     {\hbox{$>$}\nointerlineskip\hbox{$\sim$}}}}

% \textwidth 6.1in
% \textheight 8.25in
% \oddsidemargin 0.24in
%\evensidemargin -0.5in
%\topmargin -0.25in

\begin{document}

\title{Software Process for Multicomponent Multiphysics
  Codes}

\tableofcontents
%\include{foreword}
%\include{preface}
%\listoffigures
%\listoftables
%\include{contributor}
%\include{symbollist}

\mainmatter

\chapterauthor{Anshu Dubey}{ANL}
\chapterauthor{Katherine Riley}{ANL}
\chapterauthor{Katie Antypas}{LBL}




% \begin{abstract}
% Multiphysics multicomponent scientific codes have increasingly been 
% adopting software processes derived from outside the scientific
% domain. The driving force behind adoption is usually the realization
% that without using software engineering practices, the development,
% verification and maintenance of code becomes intractable. However,
% many software best practices need modification and/or customization.
% Sometimes the inherent physics of scientific applications require
% different software methodologies, while other times a premium is
% placed on performance rather than architecture.  Still others are more
% sociological and due to the type of institutions that are the typical
% homes for such codes. The challenges for scientific applications range
% from their architecture to the process for their maintenance and
% growth. For example, sometimes modularity and encapsulation principles
% are challenged by the need to tightly couple physics solvers to data
% structures.  Scientific codes are designed to explore phenomena that
% are not very well understood, so their verification strategies have to
% be customized. % The components of scientific codes must also consider
% % accuracy and stability, and interoperability between components must not
% % violate physical constraints. These constraints are particularly challenging for
% % numerical software because a wrong answer may not be detected to be wrong
% % because of the exploratory nature of the problems being addressed.  Finally, as
% % the research in the field evolves, codes may have to undergo radical
% % changes in base algorithms to include the newly acquired knowledge. And
% % the metrics of success are harder to define. 
% The institutional challenges in developing and maintaining scientific codes arise because the 
% research institutions in which they are developed often have an unreliable
% funding model and a transient developer population of graduate
% students. 
% % There are
% % rarely enough resources  to have any redundancy in expertise which makes
% % procedures like code review difficult. 
% Intellectual ownership, code 
% distribution, attribution of credit and contribution policies can all
% become thorny issues because there is often a lack of communication
% and sometimes even distrust among the stakeholders. 
% % Documentation
% % takes on a different kind of urgency because of transient expertise,
% % and yet it is notoriously difficult to persuade the 
% % developers in the field to spend their limited time providing
% % exhaustive documentation. 
% We elaborate on the above challenges and
% how they were addressed in FLASH, a code that has been extremely
% successful as a community code for several research communities. We
% outline specific solutions used by FLASH, and discuss their possible
% generalizations that are usable by other similar software 
% efforts. In particular, we address the issues related to software
% architecture and modularization, design of a testing regime,
% unique documentation needs and challenges, use of versioning system 
% for managing projects, and the tension between intellectual property
% management and open science. 
% \end{abstract}

\chapter{Software Process for Multiphysics Multicomponent Codes}
\section {Introduction} 
\label{sec:introduction}
Computational science and engineering (CSE) communities develop
complex applications to solve scientific and engineering challenges,
but these communities have a mixed record of using software
engineering best practices. Many codes developed by CSE communities
adopt standard software practices when the size and complexity of an
application becomes too unwieldy to continue without them
\cite{cc2012}. The driving force behind adoption is usually the
realization that without using software engineering practices, the
development, the verification and the maintenance of applications can become
intractable. As more codes cross the threshold into increasing
complexity, software engineering processes are being adopted from
practices derived outside of the scientific and engineering domain.
State-of-the-art for software engineering practices in CSE codes often lags
behind that in the commercial software space. 
% \comment{KA NOTE: Do we have any
% reference for this claim? AD: I don't know of one, though this seems
% to be the consensus.  Let's just soften a little then.  I added "often"}
There are many reasons for it, first, is
that not all software engineering practices can actually be adopted by
the CSE code developers.  Secondly, there is often little funding for
code development and maintenance. And finally, there is very little
research in the software engineering community targeted to specific
needs of CSE codes. 

Many software best practices are not well-suited for CSE codes without
modification and/or customization.  In particular, multiphysics and
multicomponents codes that run on the large High Performance Computing
(HPC) platforms have specific requirements.  In some cases, the
inherent physics of scientific applications require different software
methodologies.  In other cases, a large premium is placed on
performance, rather than code architecture, making it necessary to
sacrifice some known software engineering best practices.  Still
others are more sociological because codes may be developed by domain
scientists and their graduate students who have different priorities
and scientific goals.  
%and due to the type of institutions where
%such codes are developed.  
The challenges in developing scientific and engineering applications range
from data dependencies, to architectural trade-offs, to the process for their maintenance and
growth. 
% \comment{KA NOTE: see if you like what I re-wrote in last sentence.}
The standard practices adopted by the CSE codes include
repositories for code version control, modular code design, licensing
process, regular testing, documentation, release and distribution
policies, and contribution policies. Less frequently used practices
include code-review, code-deprecation, and adoption of development
methodologies such as Agile \cite{agile}. 
%Agile developmenet methods \cite{}
%tend to have limited usefulness in the CSE lifecycle.
% only a 
%select subset of those have met with success.  (KA NOTE: NEED REF) 
The degree of adoption and
sophistication in using software engineering practices varies among teams. Many of
the reasons for lower penetration of more formal software engineering practices are discussed in section
\ref{sec:instChallenges}. Even among the widely adopted
practices, most are modified and customized by the developers for
their own needs. The next few sections outline the challenges that
are either unique to, or are more dominant in this domain than
elsewhere.  


% For example,
% sometimes modularity and encapsulation principles are challenged by
% the need to tightly couple physics solvers to data structures.
% Scientific codes are designed to explore phenomena that are not very
% well understood, so their verification strategies have  to

\section{Lifecycle}
\label{sec:lifecycle} 
Scientific software is designed to model some phenomena in the
physical world. The phenomena may be at microscopic level, for example
protein folding, or at extremely large scales, for example galaxy cluster
mergers.  In some applications multiple scales are modeled.  (The term 'physical' used here includes chemical and
biological systems since physical processes are underlying building
blocks for those systems too.) The physical characteristics of the systems being modeled are
translated into mathematical models that are said to describe the
essential features of the behavior of the system being
studied. These equations are then discretized, and numerical algorithms
are used to solve them. This process requires diverse expertise and
adds many stages in the development and lifecycle of scientific
software that may not be encountered elsewhere. 
\comment{KA NOTE: Do we have any thing to support this claim?  If not
  I would just delete the sentence as it doesn't help our argument.} 
\response{AD: Does that read better ?}

\subsection{Development Cycle}
\label{sec:dev-cycle}
For scientific simulations, modeling begins with equations that describe the
general class of behavior to be studied, for example the Navier-Stokes
equations describe the flow of compressible and incompressible
fluids, and Van-der-vaal equations describe force interactions among
molecules in a material. There may be more than one set of equations
if there are behaviors that are not adequately captured by one set.
In translating the model from mathematical representation to
computational representation two processes go on simultaneously,
discretization and approximation. One can argue that discretization is,
by definition, an approximation because it is in effect sampling
continuous behavior where information is lost between sampling
intervals. This loss manifests itself as error terms in the discretized
equations, but error terms are not the only
approximations. Depending upon the level of understanding of specific
sub-phenomena, and available compute resources, scientists also 
use their judgement to make other approximations. Sometimes, to focus on a
particular behavior, a term in an equation may be simplified or may be even completely
dropped. At other times some physical details may be dropped
from the model because they are not understood well enough by the
scientists.  Or the model itself may be an approximation.  

The next stage in developing the code is finding the appropriate
numerical methods for each of the models. Sometimes good methods exists that
can be used ''as-is''.  Other times, they may need to be customized, or new
methods may need to be developed. There may need to be validation of
the method's applicability to the model if the method is new or
significantly modified. Unless an implementation of the method is
readily available as a third party software (stand-alone or in a
library), it has to be implemented and verified. It is at this stage
that the development of a CSE code begins to resemble that of general
software. The numerical algorithms are specified, the semantics are
understood, and they need to be translated into executable
code. Figure \ref{Fig:dev-cycle} gives an example of the development
cycle of a multiphysics application modeled using partial differential
equations. 

\begin{figure}[!t]
\centering
\includegraphics[ width=4.0in]{CSE-design}
\vskip -0.25in
\caption{Development cycle of modeling with partial differential equations}
\label{Fig:dev-cycle}
\end{figure}



\subsection{Verification and Validation}
\label{sec:vandv}
% There are many stages in the development cycle of scientific software 
% where errors can be introduced. 
% Many of the errors introduced in one
% stage have no correlation with those in other stages. A good verification and validation
% methodology will exploit this knowledge .  
\comment{KA NOTE: I don't know what the last sentence means...  can
  you expand a bit?  It pretty vague.}
\response_AD: Upon re-reading it seemed redundant, so I removed the
whole argument}
The terms verification and
validation are often used interchangeably, but to some communities have specific definitions.  
In one narrow definition, validation, ensures that the
mathematical model correctly defines the physical phenomena, while
verification makes sure that the implementation of the model is
correct. In other words, a model is validated against observations or
experiments from the physical world, whereas a model is verified by
other forms of testing.   Other definitions give broader scope to 
validation. For example, validation of a numerical
method may be constructed through code-to-code comparisons, and its
order can be validated through convergence studies. Similarly, the
implementation of a solver can be validated against an analytically
obtained solution for some model if the same solver can be
applied and the analytical solution is also known, though this is not
always possible.  Irrespective of  specific definitions, what is true is that
correctness must be assured at all the stages from model to
implementation.  

There are many degrees of freedom in the process of deriving a
model as discussed in the previous section, therefore, the validation of the
model must be carefully calibrated by scientific experts. Similarly,
verification of a numerical method requires applied math expertise
because the method needs verification of its stability, accuracy and
order of convergence, in addition to correctness. Numerical methods
have their own error analysis because of approximations and many of
these methods are themselves objects of ongoing research, so their
implementation may need modifications from time to time. Whenever
this happens, the entire gamut of verification and validation needs to
be applied again. This is an instance of a particular challenge in the
CSE software where no amount of specification is enough to hand the
implementation over to software engineers or developers who do not have domain or method knowledge. A close
collaboration with applied mathematicians and method developers is necessary because the process has to be iterative with
scientific judgement applied at every iteration. 

One other unique verification challenge in CSE software is the
consequences of finite machine precision of floating point
numbers. Any change in compilers, optimization levels, and even order
of operations can cause numerical drift in the solutions. Especially
in applications that have a large range of scales, it can be extremely
difficult to differentiate between a legitimate bug and a numerical
drift. Therefore, relying upon bitwise reproducibility of the solution is
rarely a sufficient method for verifying the continued correctness of
an application behavior. Robust diagnostics (such as statistics or
conservation of physical quantities) need to be built into the
verification process.  This issue is
discussed in greater detail in chapter \ref{chp:testing}.

\subsection{Maintenance and Extensions}
\label{sec:maintain}
In a simplified view of software lifecycle, there is a design and development phase,
followed by production and maintenance phase. \comment{KA NOTE:  I'm
  not sure we want to make this claim.  I've softened it a bit.
  Commercial codes are constantly releasing new features....}
\response{AD: OK, I have further diluted it}   Even in well engineered
codes this simplified view typically applies only to the
infrastructure and API's which have a distinct development phase which
has limited spill into the remainder of the lifecycle. The numerical
algorithms and solvers can be in a continually evolving state
reflecting the advances in their respective fields.  
\comment{KA NOTE: NEED REF - can we claim 'most successful' here?} 
\response{AD: see if you like the revised version'}
The development of CSE software is
usually responding to an immediate scientific need, so the codes get
employed in production as soon as a minimal set of computational
modules necessary for even one scientific project are
built. Similarly, the development of computational modules almost
never stops all through the code lifecycle because new findings in science
and math almost continuously place new demands on them. The additions
are mostly incremental when they incorporate new findings into an
existing feature, they can also be substantial when new capabilities
are added. The need for new capabilities may arise from 
greater model fidelity, or from trying to simulate a more complex
model. Sometimes a code designed for one scientific field may have
enough in common with another field that capabilities may be added to
enable it for the new field.   

Whatever may be the cause, co-existence of development and
production/maintenance phases is a constant challenge to the code
teams. It becomes acute when the code needs to undergo major version
changes. The former can be managed with some repository
discipline in the team coupled with a solid testing regime. The latter
is a much bigger challenge where the plan has to concern itself with
questions such as how much backward compatibility is suitable, how
much code can go offline, and how to reconcile ongoing development in
code sections that are substantially different between versions.
FLASH's example in section \ref{sec:FLASHSoftwareProcess} describes
a couple of strategies that met the conflicting needs of developers and
production users in both scenarios. Both required co-operation and
buy-in from all the stakeholders to be successful. 

\subsection{Using CSE Software}
\label{sec:using}
There is a fundamental requirement from the users of scientific
software that rarely comes into play for users of other kinds of
software. For good results, the users of scientific software cannot
treat it as a black box. They must understand the models well. They
must also know and understand the range of applicability of numerical
algorithms to their physical regimes, and also the accuracy and
stability behavior of the algorithms. It is very possible to apply the
methods in inappropriate ways and obtain scientifically useless
results. Even worse, one may obtain wrong results and not even know
that they are wrong. 
This is because some phenomena are very sensitive
to perturbations. If one applies a method without sufficient
resolution, the perturbations may be filtered out and the outcome of
the simulation may be physically valid while being completely wrong
for the phenomenon being studied. Similarly, sometimes equations have
mathematically valid but physically invalid solution. A badly applied
numerical scheme may converge to such a solution. Even though in this
situation it becomes obvious that the solution is not right, it may
happen after significant wasteful use of resources.
These are some of the reasons that also play a role in tendency of scientific
codes to do strict gatekeeping for contributions,  and mostly operate
in the cathedral mode.  

\section{Domain Challenges} 
\begin{itemize}
\item diverse algorithms - different data layout needs
\item less logical complexity more numerical complexity - hard to achieve data locality 
\item robustness and stability as important as accuracy
\item physics is messy - encapsulation can be challenging
\item unit testing insufficient, not even always possible
\item integrated and system level testing very critical
\item performance portability important
\item no replication of expertise and great depth in expertise is needed at times
\end{itemize}


\section{Institutional Challenges}
\label{sec:institutional-challenges}
Many adaptations in software engineering for scientific applications
described in the previous section pertain to software design and
testing. A number of challenges also arise because of the kind of
organizations and the research communities where these codes are
developed. The most crippling and pervasive 
challenge faced by scientific codes in general, and multi-physics codes in
particular, is that funding for software development and maintenance is difficult to attain. 
There is evidence that when software is designed well
it pays huge dividends in scientific productivity from the
small number of projects that secured such funding for software
infrastructure design. Examples include community codes such as NAMD
\cite{phillips2005scalable}, Amber \cite{case2014amber} and Enzo \cite{Enzo2013} which are used by
significant number of users in their respective communities. More
persuasive case can be made by a handful of codes such as FLASH \cite{Dubey2009, Dubey2008},
Cactus \cite{blazewiczphysics} and Uintah \cite{TGRID10,uintah2} that were built for one community,
but have since expanded their capabilities to serve several other
communities using a common infrastructure.

Even with this evidence it remains difficult to obtain
funding for investment in software engineering best practices.
Available funding is most often carved out of scientific goal 
oriented projects that have their own priorities and time-line. This
model often ends up short-changing the software engineering.
The scientific output of applications is measured in terms of publications, which in
turn depend upon data produced by the simulations. Therefore, in a
project driven purely by scientific objectives, the short-term science
goals can lead to situations where quick-and-dirty triumphs over
long term planning and design. The cost of future lost productivity
may not be appreciated until much later when code base has  grown too
large to remove its deficiencies in any easy way.  Software
engineering is forcibly imposed on the code, which is at best a
band-aid solution. 

Another institutional challenge in developing good software
engineering practices for scientific codes is training students and staff to
use the application properly. Multi-physics codes require a broad range
of expertise in domain science from their developers, and software
engineering skills is an added requirement.  Often experts in a domain
science who develop scientific codes are not trained in software engineering
and many learn skills on the job through reading, or talking to
colleagues \cite{hannay2009,Nguyen-Hoan}. Practices are applied as they understand them, usually
picking only what is of most importance for their own development.
This can be both good and bad. Good because it sifts out the unnecessary aspects of SE
practice, and bad because it is not always true that the sifted out
aspects were really not necessary. It might just be that the person
adopting the practice did not understand the usefulness and impact of
those aspects.

Institutional challenges also arise from scarcity and stability of
resources apart from funding. The domain and numerical algorithmic
expertise is rarely replicated in a team developing the multi-physics
scientific application.  Even otherwise, deep expertise in the domain may be
needed to model the phenomenon right, and that kind of expertise is
relatively rare. Then there is the challenge of communicating the
model to the software engineer, if there is one on the team, or to
team members with some other domain expertise. It requires at least a
few developers in the team who can act as interpreters for various
domain expertise and are able to integrate them. Such abilities take a
lot of time and effort to develop, neither of which are
easy in academic institutions where these codes are typically
organically grown. The available human resources in these institutions
are post-docs and students who move on, so there is little retention of
institutional knowledge about the code.  A few projects that do see
the need for software professionals struggle to find ways of funding
them or providing a path for their professional growth. 

The above institutional challenges are among the reasons why it is
hard and often even undesirable to adopt any set software development
methodology in scientific application projects. For example, the principles
behind the agile manifesto apply, but not all the formalized processes
do. Agile software methods \cite{agile} are lightweight evolutionary development
methods with focus on adaptability and flexibility, as opposed to
waterfall methods which are sequential development processes where
progress is perceived as a downward flow \cite{waterfall}. 
Agile methods aim to deliver working software as early as possible
within the lifecycle and improve it based upon user feedback and
changing needs. These aims fit well with the objectives of scientific
software development as well. These codes are developed by
interdisciplinary teams where interactions and collaborations are
preferred over regimented process. The code is simultaneously
developed and used for science, so that when requirements change there
is quick feedback.  For the  same reason, the code needs to be in
working condition almost all the time. However, scarcity of resources
does not allow the professional roles in the agile process to be
played out efficiently.  There is no clear separation between the
developer and the client, many developers of the code are also
scientists who use it for their research.  Because software
development goes hand-in-hand with research and exploration of
algorithms, it is impossible to do either within fixed
time-frames. This constraint effectively eliminates using agile
methods such as {\em sprints} or {\em extreme programming}
\cite{carver2007software}. The waterfall model is even less useful
because it is not cost-effective or even possible to have a full
specification ahead of time. The code has to grow and alter
organically as the scientific understanding grows, the effect of using
technologies are digested and requirements change. A reasonable
solution is to adopt those elements of the methodologies that match
the needs and objectives of the team, adjust them where needed, and
develop their own processes and methodologies where none of the
available options apply.   

Due to the need for deep expertise, and the fact that the developer of a
complex physics module is almost definitely going to leave with
possibly no replacement, documentation of various kind takes on a
crucial role. It becomes necessary to document the algorithm, the
implementation choices, and the range of operation. The generally
preferred practice of writing self explanatory code helps, but does
not suffice. To an expert in the field, who has comprehensive
understanding of the underlying math, such a code might be accessible
without inline documentation. But not to non-experts (i.e. from
another field or a software engineer in the team if there is one) who
may have reasons to look at the code. For longevity and
extensibility, a scientific code must have inline documentation
explaining the implementation logic, and reasons behind the
choices made.   

\subsection*{Key Insights}
\label{institutional-insights}
\begin{itemize}
\item The benefits of investment in software design or process are not
appreciated, and the funding model is not helpful in promoting them either. 
\item Development requires interdisciplinary teams with good
communication, which is difficult in academic institutions. 
\item Methodologies get better foothold if they are flexible and adapt
to the needs of the development team
\item Developer population is transient, detailed inlined documentation is
necessary for maintenance.
\end{itemize}
\section{Case Study: The FLASH Code}


\subsection{Code Design}
From the outset FLASH was required to have composability because the
simulations of interest needed capabilities in different permutations
and combinations. For example, most simulations needed compressible
hydrodynamics, but with different equations of state. Some needed to
include self-gravity while others did not. An 
obvious solution was to use object-oriented programming model with
common API's and specializations to account for the different
models. However, the physics capabilities were mostly legacy with F77
implementations. Rewriting the code in an object oriented language was
not an option. A compromise was found by exploiting the unix directory
structure for inheritence, where for a code unit the top level
directory defined the API and the subdirectories contained the
multiple alternative implementations of the API.  Meta-information
about the role of the directory level in the object oriented framework
was encoded in a very limited domain-specific language (configuration
DSL). The meta-information also included state and runtime variables
requirements, dependences on other code units etc. A ``setup tool''
parsed this information to configure a consistent ``application''. The
setup tool also interpreted the configuration DSL to implement 
inheritence using the directory structure. For more details about
FLASH's object oriented framework see \cite{Dubey2009, Fryxell2000}.   

FLASH design is aware of the need for separation of concerns and
achieves it by separating the infrastural components
from physics. The abstraction that permits this approach is very
well known in CSE, that of decomposing a physical domain into
rectangular form, and surrounding each of the subdomains with halo
cells copied over from the surrounding neighborin subdomains. To the
physics whole domain is not distinguishable from a sub-domain. 
It is also important not to let any of the  physics own the state
variables. They are owned by the infrastructure, which 
decomposes the domain into blocks. A further separation of concern
takes place within the units handling the infrastructure, that of
isolating the parallel aspects from bulk of the code. Parallel
operations such as ghost cell fill, refluxing or regridding have
minimal interleaving with state update in the blocks from application
of physics operators. To distance the solvers from their parallel
constructs, the required parallel operations provide an API with
corresponding functions implemented as a subunit. The implementation
of numerical algorithms for physics operators is sequential,
interspersed with access to the parallel API as needed. This does
impose bulk synchronous communication model on the code, and may need
to be modified.

Minimization of data movement is achieved by letting the state be
completely owned by the infrastructure modules. The dominant
infrastructure module is the {\em Eulerian} mesh, owned and managed by
the {\em Grid} unit. The physics modules query the {\em Grid} unit
for the bounds and extent of the block they are operating on, and
get a pointer to the physical data. This arrangement works in most
cases, but gets tricky where  the data access pattern does not conform
to the underlying mesh. An example is any physics dealing with
Lagrangian entities (LE's). They need a different data structure, and
the movement of data has nothing in common with the way the data moves
on the mesh. The added difficulty is that the entities do need to
interact with the mesh, so physical proximity of the corresponding
mesh cell is important in distributing the LE's. This is one of the
examples of unavoidable lateral interaction between modules. In order
to advance, LE's need to get some field quantities from the mesh and
then determine their new locations internally. In some applications
they have to apply near- and far-field forces, and in some
applications they have to pass some information along to the mesh. And
after advacing in time they may need to be redistributed. FLASH solves
this conundrum through keeping the LE data structure extremely simple,
and using argument passing by reference in the API's. The LE's are
attached to the block in the mesh that has the nearest cell, an LE
leaves its block when its location no longer overlaps with the
block. Migration to the new block is an independent operation from
everything else that goes on with the LE's In FLASH parlance this is
the Lagrangian framework (see \cite{Dubey2012} for more details). The
combination of {\em Eulerian} and {\em Lagrangian} frameworks that
interoperate well with one another has succeeded in meeting all the
performance critical data management needs of the code so far. 

\subsection{Software Process}
The software process of FLASH has evolved organically with the growth
of the code. For instance, in the first version there was no clear
design document, the second version had a loosely implied design
guidance, whereas the third version documented the whole design
process. The third version also published the developer's guide which
is a straight adaptation from the design document. Because of multiple
developers with different production targets versioning repository was
introduced early in the code life cycle. The repository used has been
SVN since 2003, though its branching system has been used in some very
unorthodox ways to meet peculiar needs of the Flash Center. Unlike
most software projects where branches are kept for somewhat isolated
development purposes, FLASH uses branches also to manage multiple
ongoing production projects. This particular need arose because at a
point there were four different streams of production simulations going
on simulataneously in three different scientific domains. All projects
needed some stable code from the trunk, but also needed some
development flexibility in their own branches. And it was understood
that very little of the code added to the branches will make its way
back into the repository. This would have been easy with the
distributed repositories that have become available relatively
recently, but it is difficult in SVN. This was accomplished by turning
the trunk into essentially a merge area, with a separate {\em
  production} branch becoming the base for forward code merges to the
individual projects. The path was tagged-trunk => production =>
projects => merge into trunk => tag trunk when stabilized. Note that
the production branch was never allowed a backward merge to avoid the
possible inadvertent breaking of code for one project by another one.

\subsection{Policies}
\section{Generalization} : 
% this section will discuss those aspects of FLASH solutions that are generalizable
\label{sec:generalization}

Not all of the solutions described in the earlier sections for CSE
specific challenges are generalizable to all scientific software, but
the vast majority of them are. This is borne out by the fact that at a
workshop on community codes in 2012 \cite{}, all represented codes
had nearly identical stories to tell about their motivation for
adopting software engineering practices and the ones that they
adopted. This was true irrespective of the science domains these codes
served, the algorithms and discretization methods they used and
communities they represented. Even their driving design principles
were similar at the fundamental level though the details differed. The
codes represented state-of-the-art in their respective communities in
terms of both model and algorithmic research incorporated and the
software engineering practices. Note that these are the codes that
have stood the test of time and won the respect in their respective
communities. They are widely used and supported, and have more
credibility for producing reproducible reliable results than smaller
individualistic efforts. Therefore, it is worthwhile to discuss
those practices in this chapter. At a minimum they provide a snapshot
of the state of large scale computing and its dependence of software
engineering in the era of relatively uniform computing platforms. 

One practice that is universally adopted by all community codes and
other large scale codes is versioning repositories. That is worthy of
mention here because even this practice has not penetrated the whole
computational science community. There are many small projects
that still do not use versioning, though their number is steadily
decreasing. Other common practice is that of licensing for public use
and most codes are freely available to download along with their
source. Testing is also universal, though the extent and methodologies
for testing vary greatly. A general verification and validation regime
is still relatively rare, though regression testing is more
common. Unit tests are less common than integration tests and
bounded-change tests. Almost all codes have user level documentation
and user support practices in place. They also have well defined code
contribution policies. 

Another feature that stands out is the broader design philisophy of
all multiphysics codes. Every code exercises separation of concerns
between mathematical and structural parts and between sequential and
parallel parts. In almost all cases this separation is dictated by the
need to reduce complexity for efforts needing specific
expertise. Also, all the codes have basic backbone frameworks which
orchestrate the data movement and ownership. This is
usually driven by the need for maintenance and flexibility. And where
it is realized well it provides extensibility - the ability to add
more physics and therefore greater capabilities and fidelity in the
models being computed. Majority of frameworks are component based with
composability of some sort. This is because different models need
different capability combinations. Most codes use self-describing IO
libraries for their output to facilitate the use of generally
available analysis and visualization tools. 

The degree to which teams from vastly different scientific domains
producing community codes have arrived at essentially similar
solutions is remarkable. It points to a possibility that seemingly
diverse problems can have a uniform solution if they are trying to
achieve similar objectives. For the codes highlighted in this section,
the objectives were capabilities, extensibility, composability,
reliability, portability and maintainability. They were achieved
through design choices concious of trade-offs, most often with raw
performance that individual components or specific platforms were
capable of. The lesson here is that similar objectives can yield a
general solution even if there is great diversity in the details of
the individual problem. It is not beyond the realm of possibility that
similar generalized solution will emerge for the next generation
software faced with heterogeneous computing described in the next
section.

\section{Additional Future Considerations}: 
%How the software, design and policies might need to change in Future.
\label{sec:future}

An aspect of software design that is a unique requirement of
the scientific domain is fast becoming a great challenge - performance
portability. In the past, machine architectures were fairly uniform
across the board for large stretches of time.  The first set of
effective HPC machines in routine use for scientific computing were
all vectors machines. They later gave way to parallel machines with
{\em risc} processor as their main processing element. A code written
for one machine of its time, if portable, would have reasonable
performance on most of its contemporary machines. The abstract machine
model to which the codes of the era were programming was essentially
the same for all machines of that era. It is true that wholesale changes had to occur
in codes for transitioning from vector to risc-parallel machines, but
it was a transition from one long-term stable paradigm to another
long-term stable paradigm.  In addition, the codes were not as large as the
multi-physics codes of today. So although the transitions took time, the
codes that adapted well to the prevailing machine model thrived for
several years.  

The computing landscape is undergoing significant changes. Now there are machines
in the pipeline that have deep enough architectural differences among
them that one machine model cannot necessarily describe their behavior.  A look at the top supercomputers in the world shows a variety of architectures from accelerator models, to many-core systems.  Even though many different vendors are moving to architectures with lighter and smaller cores, the different cache and memory hierarchies on these systems make portability across architectures difficult.  In addition, the lack of a high performing, common programming model across architectures poses an even greater challenge for application developers.  And, because
the codes are significantly larger than they were during the last
architecture paradigm shift, the transition will be even more challenging.  More importantly, some aspects of the
challenges are not unique to the large multi-physics codes. Because the
deep architectural changes are occurring at the level of nodes that
will go into all platforms, the change is ubiquitous and will
affect everyone. Portability in general, and performance
portability in particular, is an issue for everyone. At this writing
the impact of this paradigm shift is not fully understood. Means of
combating this challenge are understood even less. There is a general
consensus that more programming abstractions are necessary not just
for the extreme scale, but also for small scale computing. The unknown
is which abstraction or combination of abstractions will deliver the
solution. Many solutions have been proposed, for example \cite{PADAL14} (also
see \cite{IDEAS} for a more comprehensive and updated
list). Of these, some have undergone more testing and exercise under
realistic application instances than others. Currently, no approach has been shown to provide a general solution that can be broadly
applicable in the ways that optimizing compilers and MPI were
in the past. This is an urgent serious challenge facing the scientific
communities today, future viability of scientific codes depends upon
significant help from software engineering expertise and motivation
within the community. 


\bibliographystyle{plain}
\bibliography{SEforCSE}
\end{document}




