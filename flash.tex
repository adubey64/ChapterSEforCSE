\subsection{Code Design}
In a CSE software design, separation of concerns is of utmost
importance. Orthogonalizing expertise requirements into different code
components allows developers to focus on what they know best.  
Following that principle, FLASH separated the parallel aspects of the
code from the numerics. The solvers are essentially sequential, except
for a few synchronization points which are encapsulated in single
function calls. The abstraction that permits this approach is to view
a block with its surrounding halo cells as the basic computation
unit. The interaction between blocks, i.e. for ghost cell fill, refluxing
or regridding are all hidden from the numerical solvers. 
To distance the non-explicit solvers from their parallel constructs,
the required parallel operations provide an API with corresponding
functions implemented as a subunit within the Grid  unit. The Grid
unit manages the mesh and also houses those 
components of solvers that require heavy interaction with the
mesh. Blocks also provide coarse grain thread based parallelism where
necessary. 

Another design choice from the early version that has persisted 
until today is to use parallel I/O libraries with self-describing data
formats. Porting the code to new HPC platforms would have become
a recurring resource drain without this choice. The ability to develop
general analysis tools and use third party tools has been an added
benefit. However, these benefits were not obvious in the early years
of the code. FLASH had adopted the parallel I/O libraries before they
had reached maturity or were supported on all platforms. I/O was
consistently the barrier to early access on new HPC platforms because 
the libraries did not immediately become available. The libraries are
now very stable and they get on the new platforms in roughly the same
time frame as compilers do.  At the scale at which FLASH is used for
production now, any other form of I/O would cause unacceptable delay
in scientific analysis of the data produced by the simulations.  

Many architectural elements, such as inheritance, customization, and
the existence of multiple alternative implementations, have stayed the same
throughout the evolution of the code. Some others, such as data
management and scoping and encapsulation of code components, changed
over time, while several others, such as the formalization of unit
architecture and the unit testing framework, were introduced
later. However, the philosophy of the FLASH architecture is
essentially what was laid down in the early versions
\citep{Fryxell2000}.  A description of the current architectural 
framework is described in greater detail in \citet{Dubey2009}. 

The most influential design choice made for FLASH3 was to dispense
with backward compatibility with earlier versions. This choice was
made easier by the deep changes in PARAMESH and HDF at roughly the
same time. It led to a three-fold benefit: (1) it was possible to make
fundamental changes to the structure of the code units and their data
management, (2) we could re-examine the lateral interactions between
units dictated by physics in ways that enabled unit encapsulation and
therefore extensibility, and (3) it permitted pruning of the source tree
where needed to help unclutter the code infrastructure. 

FLASH3 also focused on eliminating limitations to extensibility that
came from  the specifications of the FLASH's mesh being tied very
closely to those of PARAMESH. For example, all blocks have the same
number of cells in PARAMESH, therefore, FLASH up to version 2 fixed
number of cells per block at configuration
level. Similarly, the order of indices in the solution data arrays
followed the PARAMESH mode, where field variables were the first index
in a Fortran 90 array. These assumptions effectively eliminated
the possibility of using other available AMR packages. 
With these limitations removed, other block structured AMR packages
can be considered for inclusion in FLASH. For example, 
the release of version 4 includes Chombo \citep{Chombo2009} as an
alternative AMR mesh within FLASH. Several other modifications
to code architecture and functionality used in FLASH3, 
such as formalization of the code unit architecture, are described in
\citet{Dubey2009}.  The architecture of FLASH3 has proved to be robust
enough that it has not required any major infrastructure revision for
adding the new capabilities for HEDP and FSI. Even though FLASH4 is a
major version change, it inherited its architecture and software
engineering practices from FLASH3. It is only the advent of a major
shift in hardware architecture that is proving to be a challenge and
is likely to result in fundamental refactoring of the code
infrastructure in future.

\subsection{Software Process}
\subsection{Policies}