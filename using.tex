\label{sec:using}
There is a fundamental requirement from the users of scientific
software that rarely comes into play for users of other kinds of
software. For good results, the users of scientific software cannot
treat it as a black box. They must understand the models well. They
must also know and understand the range of applicability of numerical
algorithms to their physical regimes, and also the accuracy and
stability behavior of the algorithms. It is very possible to apply the
methods in inappropriate ways and obtain scientifically useless
results. Even worse, one may obtain wrong results and not even know
that they are wrong. 
This is because some phenomena are very sensitive
to perturbations. If one applies a method without sufficient
resolution, the perturbations may be filtered out and the outcome of
the simulation may be physically valid while being completely wrong
for the phenomenon being studied. Similarly, sometimes equations have
mathematically valid but physically invalid solution. A badly applied
numerical scheme may converge to such a solution. Even though in this
situation it becomes obvious that the solution is not right, it may
happen after significant wasteful use of resources.
These are some of the reasons that also play a role in tendency of scientific
codes to do strict gatekeeping for contributions,  and mostly operate
in the cathedral mode.  