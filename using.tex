\label{sec:using}
There is a fundamental requirement from the users of scientific
software that rarely comes into play for users of other kinds of
software. For obtaining valid scientific results the users must 
have a basic understanding of the phenomena being modeled and the
approximations, and they must know which questions can be validly 
addressed by these models. For example, Eulerian equations of
hydrodynamics do not account for viscosity. If one is modeling a fluid
where viscosity is important, one should not use a code that uses
Eulerian equations to model the fluid. Similarly, they must know and
understand the valid regimes for applicability of numerical
algorithms, and also the accuracy and stability behavior of the
algorithms. For example a numerical method that can resolve a smooth
fluid flow only will fail if there are discontinuities. Similarly Fast
Fourier Transform (FFT) methods work really well for periodic
boundaries. For all other boundaries they either need some additional
processing, or are not applicable. Also, in order to use the FFT
method the user must know what is the highest mode they are
resolving. If enough terms are not used in the Fourier series
approximation the computed result may filter out important information
and lead to wrong results.

% It is very possible to apply the  
% methods in inappropriate ways and obtain scientifically useless
% results. Even worse, one may obtain wrong results and not even know
% that they are wrong. 
% This is because some phenomena are very sensitive
% to perturbations (KA NOTE: in the model?  in what?  need to explain
% perturbations in this context.) 
% If one applies a method without sufficient
% resolution, the perturbations may be filtered out and the outcome of
% the simulation may be physically valid while being completely wrong
% for the phenomenon being studied. 
Similarly, sometimes equations have mathematically valid but
physically invalid solution. A badly applied numerical scheme may
converge to such a non-physical solution. At best any of the above
situations may lead to a waste of computing resources if the defect in
the solution is detected. At worst they may lead to wrong scientific
conclusions being drawn. Some in the scientific community even argue
that those who have not written at least some basic version of their
own code for their own problem should not be using other's public or
community code. Though that argument goes too far, it embodies a
general belief in the scientific community that users of scientific
codes have a great responsibility to know their tools and understand
their capabilities and limitations. 
% Even though in this 
% situation it becomes obvious that the solution is not right, it may
% happen after significant wasteful use of resources.
These are some of the reasons that also play a role in tendency of scientific
codes to do strict gatekeeping for contributions,  and mostly operate
in the cathedral mode.  

%\comment{KA NOTE:  this is much better...}