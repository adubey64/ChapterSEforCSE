\label{sec:using}
Users of scientific software must have a basic understanding of the
models, the approximations and the numerical methods they are using to
obtain valid scientific results. They must know and
understand the valid regimes for applicability of numerical
algorithms, as well as their accuracy and stability behavior. For
example a numerical method that can resolve a smooth 
fluid flow only will fail if there are discontinuities. Similarly, in
order to use the Fast Fourier Transform (FFT) method, users must
ensure that their sampling interval resolves all modes. Failure to do
so may filter out important information and lead to wrong results.
Sometimes equations have mathematically valid but physically invalid
solutions. An inappropriately applied numerical scheme may converge to
such a non-physical solution.  

At best any of the above situations lead to a waste of computing resources if the defect in
the solution is detected. At worst they may lead to wrong scientific
conclusions being drawn. Some in the scientific community even argue
that those who have not written at least a simplistic version of the
code for their application should not be using other's code. Though
that argument goes too far, it embodies a general belief in the
scientific community that users of scientific codes have a great
responsibility to know their tools and understand their capabilities
and limitations.  
These are some of the reasons that also play a role in tendency of scientific
codes to do strict gatekeeping for contributions.

