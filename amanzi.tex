\subsection{Amanzi/ATS}
\label{sec:amanzi}
%
Amanzi and its sister code the Advanced Terrestrial Simulator (ATS), provide a good contrasting example to FLASH.
Developed starting in 2012 as the simulation capability for the Department of Energy's Environmental Management program, Amanzi solves equations for flow and reactive transport in porous media, with intended applications of environmental remediation for contaminated sites \cite{MoultonMD12}.
Built on Amanzi's infrastructure, ATS adds physics capability to solve equations for ecosystem hydrology, including surface/subsurface hydrology, energy and freeze/thaw cycles, surface energy balance and snow, and vegetation modeling \cite{PainterMW13,AtchleyPHC15}.
Amanzi was initially supported by a development team of several people with dedicated development money.
ATS was largely developed by one person, post-docs, and a growing set of collaborators from the broader community, and was supported by projects whose deliverables are ecosystem hydrology papers.

Amanzi/ATS's history makes it a good contrast to FLASH.
Developed from the ground up in C++ using relatively modern software engineering practices, there are few legacy code issues.
Unlike FLASH, Amanzi/ATS makes extensive use of ``third party libraries'', with associated advantages and disadvantages (currently Amanzi/ATS uses nearly 10k lines of cmake to build it and its libraries).
% TODO -- Anshu, is this correct?
However, they also share a lot of commonalities.
Like FLASH, version control has played a critical role in the development process, especially as developers are spread across multiple physical locations and networks.
Like FLASH, Amanzi/ATS makes extensive use of module-level and regression-level testing to ensure correctness and enable refactoring.
And like FLASH, Amanzi/ATS has found the open source strategy to be incredibly useful; in particular, the open source nature of the code has eliminated some of the natural competition between research groups at different DOE national laboratories and helped establish a growing community of users and developers.
% TODO --- check terminology to make sure consistent with Ross chapter
% TODO --- any other ties to make?  

\subsubsection{Multiphysics management through Arcos}
\label{sec:amanzi:arcos}
%
Recognizing early the wide variety of multiphysics applications that would be targeted with Amanzi/ATS, a formal multiphysics framework was designed, implemented, and adopted.
This framework, later named Arcos \cite{CoonMP16}, consists of three main components: a \emph{process tree}, a \emph{dependency graph}, and a \emph{state/data manager}.

The process tree describes the hierarchical coupling between equations and systems of equations to be solved.
Each leaf node of the tree is a single (partial) differential equation, such as conservation of mass.
Each interior node of the tree couples the children below it into a system of equations.
Every node presents a common interface to the nodes above it.
Much of the coupling of internal nodes can be automated using this interface -- sequential couplers can be fully automated, while globally implicit coupled schemes can be nearly automated (with off-diagonal blocks of preconditioners and globalization of nonlinear solvers the lone exceptions).
This representation of multiphysics models is natural to the coupled physical system, and implicitly exists in most codes; Arcos makes this explicit while providing hooks for customization to the specific system.

A second view of the system of equations is stored in the dependency graph, much like that of \cite{Notz2012}.
The dependency graph is a directed, acyclic graph (DAG) where each node is a variable (either primary, secondary, or independent), and each edge indicates a dependency.
The graph is built up from the leafs of the DAG, which are primary variables (those variables to be solved for) and independent variables (data provided to the model).  Roots of the DAG are, for instance, corrections to the primary variable formed through a nonlinear solve, or time derivatives of primary variables used in time integrators.
Between these, each interior node is a secondary variable, and consists of both data and an \emph{evaluator}, or small, stateless (functional) unit of code that stores the physics or numerics of how to evaluate the variable given its dependencies.

Finally, a state object is a glorified container used to manage data.
Through the state's interface, \emph{const} access is allowed to all evaluators, and non-\emph{const} access is strictly limited to the evaluator of that variable.

This framework, while at times seeming heavy-handed, results in several important implications from a software engineering perspective.
Here we focus on the dependency graph as used in Amanzi/ATS, and how it encourages and enables good software engineering practices.

\subsubsection{Code Re-use and Extensibility}
\label{sec:amanzi:modularity}
%
The first observation of this framework is that it results in extremely fine-grained modularity for physical equations.
Most computational physics codes are modular at the level of equations; Amanzi/ATS is modular at the level of terms in the equation.

An example illustrates the usefulness of this in multiphysics applications.
In a thermal hydrology code, the liquid saturation is a variable describing the volume fraction of pore space that is water.
This is a secondary variable, and is a function of either liquid pressure (in isothermal cases) or of liquid pressure and temperature (in non-isothermal cases).
By explicitly forming a dependency graph, terms that depend upon the liquid saturation need not know whether the model is isothermal or not.
Furthermore, there is no concern of ``order of equation evaluation'' that is common in other multiphysics codes.
As both the energy and mass conservation equations need liquid saturation, this model should be evaluated when it is first needed, but is likely not necessarily evaluated in both equations, as its dependencies have not changed.
Optimization of evaluating a model only when its dependencies have changed results in tightly coupled, monolithic physics implementations.
Often codes will have ``modes'' which reimplement the same mass conservation equation twice, once to support the isothermal case and once to support the non-isothermal case as coupled to an energy conservation equation.

By storing these dependencies in an explicit graph, the framework can keep track of when dependencies have changed, and lazily evaluate models exactly when needed.
In tightly coupled multiphysics, a dependency graph eliminates the need for an omnipotent programmer to carefully orchestrate when and where each equation term is evaluated.
As a result, a dependency graph eliminates code duplication and discourages monolithic physics code.

Furthermore, this strategy greatly improves code extensibility.
Many variations in model structure are easily implemented by writing a new evaluator and using it in an existing conservation equation.
Once past an initial developer learning curve, Amanzi/ATS developers are able to quickly adapt the code to new models.

\subsubsection{Testing}
\label{sec:amanzi:testing}
%
Testing is an extremely sensitive subject in computational software engineering -- so much so that it merits its own chapter: \ref{}.
Few CSE codes are sufficiently tested by conventional Software Engineering (SE) standards, and many CSE developers are aware of the shortcoming.
As discussed above, frequently CSE codes are limited to component-level tests, as it can be difficult to write sufficiently fine-grained unit tests.
%TODO -- is ``component-level'' the right name here?  Check with Ross chapter
SE techniques such as mocking objects are almost never practiced, as mocked objects would require nearly all of the same functionality of the real object in order to properly test the physics component.
The claim is that most physics code cannot be tested without including discretizations, solvers, meshes, and other components.

This viewpoint, however, is somewhat narrow.
Physics at the level of a differential equation cannot be tested at the granularity of a unit test.
Physics at the level of a term within an equation, however, is much easier to test.
By forcing componentization at the level of an evaluator, Amanzi/ATS allows a large portion of its physics implementation to be unit tested.
Evaluators (and their derivatives) are stateless, and therefore can be tested without additional components.
Pushing the majority of the physics implementations into evaluators and out of monolithic, equation-level classes greatly improves the code coverage of fine-grained unit tests.
Amanzi/ATS still makes extensive use of medium-grained component tests for discretizations, solvers, etc, but a significantly large portion of physics testing is done at the unit-test granularity.

Amanzi/ATS additionally maintains a large list of coarse-grained system-level tests.
%TODO -- again, be consistent with testing chapter
These test the full capability, and serve additionally as documentation and example problems for new users.
This strategy, of providing large suites of sample problems for both testing and documentation has become extremely common in the CSE community, and is widely considered a CSE best practice.
By greatly lowering the bar for new users, this collection of dual-purpose examples encourages community; in ATS, each new physics contribution must be accompanied by a new system-level test for inclusion in these test suites.
% TODO -- same as above

In Amanzi/ATS, unit and component granularity tests are automated using ctest and run sufficiently fast to be run prior to every commit.
While Amanzi/ATS does not practice true continuous integration, it is expected that all developers run this testsuite prior to committing to the main repository.


\subsubsection{Performance Portability}
\label{sec:amanzi:performance}
%
Amanzi/ATS was designed from the ground up with awareness of ongoing, disruptive architecture changes.
Performance portability in the face of these changes is an issue that is and will continue to confront all codes.
As such, Amanzi/ATS takes several strides to buffer itself from disruptive change.

First, by leveraging popular, supported, open source libraries with significant community, Amanzi/ATS is able to immediately leverage advances in these codes.
For instance, a significant portion of time is spent in inverting a preconditioner using approximate linear solvers.
By using a common interface and a wide variety of existing solver libraries through Trilinos, Amanzi/ATS is able to immediately leverage advances in any one of these libraries.

Next, by encouraging overdecomposition of physics into smaller, more heterogeneous units, Amanzi/ATS is set up to leverage task-based programming models with novel runtime systems.
While not currently implemented, using one of the several emerging ``coarse task'' runtime environments \cite{BauerTSA12,Qingyu,CharmPP} is an exciting research area for the community.

Finally, Arcos's evaluators are a functional programming concept, and are stateless functors with no side effects.
As a result, they abstract ``what is evaluated'' from ``on what data and chip is it evaluated.''
This design decision allows the implementation of multiple versions of the latter (i.e. GPU, many-core) without touching any of the former code.

